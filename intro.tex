\chapter{Themenstellung}
\label{chap:intro}

In der vorliegenden Hausarbeit widmen wir uns der Dokumentation zu einer Anwendung für Professoren und Studierende. Diese dient dazu, während der Vorlesung Pro- und Kontra-Argumente zu einem vom Professor vorgegebenem Thema zu sammeln. 
Für Studenten wird die Anwendung in Form einer Android-Application (kurz AA) verfügbar sein. Der Professor verwendet eine mit dem Beamer kompatible Webanwendung. In dieser WebApp kann er  das Thema erstellen und die von den Studierenden eingeschickten Argumente verwalten, sortieren und ordnen.
Hierbei spielen interaktive Bedienelemente zur Verwaltung der gesammelten Informationen eine zentrale Hauptrolle. 

Die Anforderungen an die App sind je nach Nutzer unterschiedlich. Der Studierende soll durch einfach verständliche Funktionen in seinem Bestreben, sich in der Vorlesung einzubringen unterstützt werden. Für ihn ist das abgeben von Argumenten die zentrale Funktion. Diese soll durch möglichst wenig Klicks erreicht werden. Der Student kann die bereits abgegebenen Argumente seiner Kommilitonen einsehen und eventuell bewerten. 
Für den Professor ist das Erstellen einer Thematik und die Auswertung, Sortierung und Besprechung aller eingeschickten Argumente die zentrale Aufgabe, bei der die App ihn unterstützen muss. Hierbei soll auch eine Regelsystem berücksichtigt werden, damit  die Studenten die Funktionen der App nicht missbrauchen können. Der Professor muss die Argumente nach Relevanz, Redundanz, Zusammengehörigkeit etc. sortieren können. Auch das Löschen einzelner Argumente soll berücksichtigt werden (z.B. bei Dopplung). Der Professor kann die Umfragen als PDF exportieren. Er kann diese PDF an alle mitwirkenden Studenten und interessierte Kollegen verschicken. 

Von technischer Seite her sollen die Themen vom Rechner / Laptop des Professors auf die Smartphones der Studenten über Bluetooth übertragen werden. Es soll einen Offline-Modus geben, mit der die Themen auch ohne Bluetooth-Verbindung von zu Hause ergänzt werden können. 
Jedes Thema bekommt vom Professor einen begrenzten Zeitraum zugewiesen. Nur in diesem Zeitraum können die Studenten Argumente zum Thema beitragen. Danach gilt das Thema als geschlossen.

Die nächste Teilaufgaben stellen sich wie folgt dar:
\begin{itemize}
\item Klärung aller Funktionen
\item Erstellen mehrere Storyboards (pro Person eines)
\item Festlegen auf ein gemeinsames Storyboard
\item Chancen und Risiken zusammentragen
\item Personas erstellen
\item Objekt-Hierarchie und Bedienelemente festlegen
\end{itemize}




