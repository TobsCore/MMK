\chapter{Themenstellung}
\label{chap:intro}

In der vorliegenden Hausarbeit widmen wir uns der Dokumentation zu einer Anwendung für Professoren und Studenten, welche dazu dient, während der Vorlesung Pro –und Kontraargumente zu sammeln, um das interaktive Mitarbeiten von unmotivierten Studenten in der Vorlesung zu stärken. Für Studenten wird die Anwendung in Form einer Android – Application (kurz AA) verfügbar sein, in welcher zu einer vom Dozenten vorgegebenen Themenstellung Argumente verfasst und abgeschickt werden können. Der Professor verwendet zur Präsentation des Ergebnisses eine mit dem Beamer kompatible Webanwendung, um die Argumente der Studenten sammeln und vorstellen zu können. Hierbei spielen interaktive Bedienelemente zur Verwaltung der gesammelten Informationen eine zentrale Hauptrolle. 

Unsere erste Aufgabe besteht nun darin, unsere Vorstellungen zur Thematik einzugrenzen und genau zu definieren.  Zum einen wird ein zentrales Verwaltungssystem benötigt, mit welchem der Dozent alle entgegenkommenden Informationen verwalten und sie an Hand von Relevanz sortieren und unter Pro bzw. Kontra zuordnen kann. Hierbei müssen die Wertigkeit des Arguments (Pro oder Kontra?), Redundanz sowie Irrelevanz berücksichtigt werden. Dies führt uns auf direktem Wege zum nächsten Punkt, der Regelung. Es muss ein durchdachtes Regelungssystem geben, um Missbrauch durch Studenten vorzubeugen und eine klare Struktur zu schaffen. Das Ziel der Anwendung ist es, durch Abwechslung die Aufmerksamkeit sonst nicht sehr aufmerksamer Studenten zu wecken und wenn möglich, ihr Interesse an der Vorlesung zu steigern, da jeder Student direkt in das Geschehen eingreifen kann und von seinen Tagträumen wieder auf den Boden der Realität zurückfindet. Das bedeutet, dass Studenten, die nie etwas zur Vorlesung beitragen, von der Anwendung anders behandelt werden müssen, als ihre Kommilitonen, welche ohnehin schon bei jeder Gelegenheit versuchen, ihren Senf dazuzugeben. 