\chapter{Analyse}
\label{chap:analyse}

\section{Chancen und Risiken}
\label{sec:chances}

Chancen und Risiken
Die App soll die Studierenden besser in die Vorlesung mit einbeziehen. Sie soll aktivierend wirken und die Kreativität der Studierenden anregen. Durch die Nutzung des Smartphones kommen auch introvertierte, unsichere und ängstliche Studierenden zu Wort. Durch die Interaktion und die Auflockerung der Vorlesung sollen unmotivierte, tagträumende Studierende dazu animiert werden, sich einzubringen. 
Der Professor bekommt die Chance, die Denkweise der Studierende besser zu verstehen und kann sehen, wo noch Verständnis-Probleme vorhanden sind. Oftmals passiert es, dass eine abgehobene Denkweise entsteht, wenn man sich auf einem Gebiet spezialisiert. Es wird schwieriger Anfänger-Probleme und Fragen nachzuvollziehen oder überhaupt Problemquellen zu entdecken. Hierbei kann die App ebenfalls helfen. Die Vorlesung wird durch die Anwendung der  App interaktiver und somit höchstwahrscheinlich auch beliebter unter den Studierenden.

Als Risiko sehen wir einen möglichen Missbrauch der App-Funktionen von Seiten der Studierende. Durch unqualifizierte, nicht zum Thema passende Beiträge kann die Vorlesung massiv gestört werden. 
Des Weiteren darf das Interface nicht zu komplex aufgebaut sein, um auch nicht-technik-affinen Benutzern einen einfachen Umgang mit der App zu gewähren. 
Es muss festgelegt werden, wie mit Unklarheiten in der Fragestellung des Professors umgegangen wird. Ein weiteres Risiko stellt die Bewertung der Studierenden dar. Hierbei muss entschieden werden, ob die Argumente anonym abgegeben werden oder ob gezeigt wird, welcher Studierende das Argument verschickt hat, um auch eventuelle Rückfragen stellen zu können. Je nach Thema ist die Anonymisierung empfehlenswert, damit die Studierenden sich eifriger einbringen und weniger Angst haben, über ihre Argumente zur Rede gestellt zu werden.

Auf technischer Seite gibt es ebenfalls Risiken. Zum einen muss die Art der Benutzerverwaltung geklärt werden. Es besteht die Möglichkeit, dass sich die Nutzer für die App einen eigenen Account anlegen müssen oder aber sich mit ihrer Hochschul- / Universitäts-E-Mail-Adresse einloggen. 
Des Weiteren ist die Verbindung zwischen den unterschiedlichen Geräten eine Schwierigkeit. Um gutes Arbeiten zu ermöglichen, muss die Verbindung permanent bestehen oder eine lokale Kopie auf dem Endgerät des Nutzers zur Verfügung stehen. Ebenfalls müssen die Nutzern passende Endgeräte besitzen.

\section{Personas}
\label{sec:personas}

\includepdf[pages={1},pagecommand={\pagestyle{fancy}}]{images/Personas.pdf}
\includepdf[pages={2},pagecommand={\pagestyle{fancy}}]{images/Personas.pdf}
\includepdf[pages={3},pagecommand={\pagestyle{fancy}}]{images/Personas.pdf}



\clearpage
\section{Fallstudien}
\label{sec:fallstudien}

\begin{enumerate}
\item Herr Meier

\paragraph{Ziel:} Herr Meier, Professor für Betriebssysteme, möchte mit seinen Studenten in der Vorlesung Vor- und Nachteile für bzw. gegen UNIX-Systeme sammeln. Sein Ziel dabei ist es, die Vorlesung etwas aufzulockern und nebenbei eine Lernerfolgskontrolle durchzuführen. 

\paragraph{Aktivitäten:}
\begin{itemize} 
\item öffnen der WebApp auf dem Laptop
\item mit Benutzername und Passwort einloggen
\item Eine neue Umfrage anlegen
\item Name der Umfrage, Vorlesung, Beschreibung, Zeitraum eintragen
\item Umfrage Speichern 
\item Umfrage überprüfen
\item Umfrage veröffentlichen
\item Ein Argument hinzufügen
\item Argumente verwalten
\end{itemize}


\item Tobias König

\paragraph{Ziel:} Tobias König sitzt in der Vorlesung und möchte sich an der Pro- \and Contra-Argument Liste seines Professors beteiligen um sich bei diesem einzuschleimen.

\paragraph{Aktivitäten:}
\begin{itemize}
\item anschalten seines Smartphones
\item downloaden der ArguMens-App 
\item Einloggen mit Benutzername und Passwort
\item Auswählen der Umfrage
\item Eingeben eines Pro-Arguments
\item Eingeben eines weiteren Pro-Arguments
\item Eingeben eines Contra-Arguments
\item eine bereits beendete Umfrage anschauen
\item zurück zur aktuellen Umfrage springen
\item ein weiteres Contra-Argument ansehen
\end{itemize}

\item Patrick Kerst

\paragraph{Ziel:} Patrick Kerst sitzt verärgert in der Vorlesung. Um seinen Frust abzulassen, möchte er Kritik in Argumente verpacken und auf die Pro-Contra-Liste setzen

\paragraph{Aktivitäten}
\begin{itemize}
\item öffnen der App
\item auswählen der gewünschten Umfrage
\item Eingeben eines Contra Arguments 
\item Eingeben eines weiteren Contra Arguments 
\item Eingeben eines weiteren Contra Arguments
\item schließen der App 
\end{itemize}
\end{enumerate}

\clearpage