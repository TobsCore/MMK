\chapter{Test}
\label{chap:test}

\section{Testvorbesprechungs-Anleitung}
\label{sec:testvorbesprechung}

Diese hier genannten Punkte sind die Anweisung für den Testleiter, wie er den Probanden auf den Test vorbereiten soll. Der Testleiter probiert diese Punkte in einer lockeren Atmosphäre zu erläutern.

\begin{itemize}
\item Dem Probanden muss klargemacht werden, dass nicht er, sondern nur unsere App getestet wird. Dabei soll uns der Proband helfen.

\item Sage dem Probanden, dass wir versuchen, Fehler an der grafischen Oberfläche zu erkennen.

\item Er soll versuchen, sich diese zu merken, wenn ihm etwas komisch vorkommt oder bestimmte Dinge nicht gefunden werden. Wir sind für jeden gefundenen Fehler an der App dankbar.

\item Erkläre dem Teilnehmer, dass er sich bei dem Test so viel Zeit lassen kann, wie er möchte.

\item Erkläre dem Teilnehmer, dass es vollkommen in Ordnung ist, den Test abzubrechen oder eine kleine Pause einzulegen, wenn er will.

\item Erkläre dem Teilnehmer, dass alle Aufzeichnungen/Daten seiner Umfrage nicht weitergegeben werden und lediglich für die Hochschule gebraucht werden.

\item Sage dem Probanden, dass wir aufschreiben, wie lange er für jede Aufgabe benötigt und, dass wir gegebenenfalls Notizen dazu machen.

\item Der Teilnehmer darf sich danach die Notizen anschauen und sagen, falls ihn etwas stört. 

\item Bitte den Teilnehmer um lautes Denken. Dadurch können bestimmte Fehler an der App, die sonst nicht auffallen, schnell gefunden werden.

\item Hole Dein Handy raus und zeige dem Teilnehmer an Hand einer Standardanwendung, wie das laute Denken funktioniert, um ihm die Angst zu nehmen. 

\item Lege zum Beispiel eine Notiz mit der App Evernote an und spiel an den Einstellungen der App herum. Kommentiere jeden Schritt, den du machst.

\item Sage dem Teilnehmer, dass er jeder Zeit fragen stellen kann, und ihm dann geholfen wird. Er soll sich nicht unter Druck gesetzt fühlen.

\item Nun stelle unser Projekt vor. Zeige dem Probanden kurz eine Übersicht der wichtigen Screens und erkläre Ihm, wofür diese zuständig sind. Er soll sich nicht „ins kalte Wasser geschmissen“ fühlen. Auch hier kann er Fragen stellen.

\item Stelle dem Teilnehmer alle Aufgaben vor, die er gleich durchführen wird

\item Haben Sie noch weitere Fragen?
\end{itemize}

Im Anschluss wird dem Probanden noch das folgende Datenschutzformular vorgelegt.


\begin{wrapfigure}{L}{0.4\textwidth}
  \vspace{-20pt}
  \begin{center}
    \includegraphics[page=1,width=0.99\textwidth]{./images/datenschutz}
  \end{center}
  \vspace{-40pt}
\end{wrapfigure}

\clearpage
\section{Nachbefragungs-Formular}
\label{sec:nachbefragung}



\begin{wrapfigure}{L}{0.4\textwidth}
  \vspace{-20pt}
  \begin{center}
    \includegraphics[page=1,width=0.99\textwidth]{./images/dozent}
  \end{center}
  \vspace{-40pt}
\end{wrapfigure}


\begin{wrapfigure}{L}{0.4\textwidth}
  \vspace{-20pt}
  \begin{center}
    \includegraphics[page=2,width=0.99\textwidth]{./images/dozent}
  \end{center}
  \vspace{-40pt}
\end{wrapfigure}


\begin{wrapfigure}{L}{0.4\textwidth}
  \vspace{-20pt}
  \begin{center}
    \includegraphics[page=3,width=0.99\textwidth]{./images/dozent}
  \end{center}
  \vspace{-40pt}
\end{wrapfigure}


\begin{wrapfigure}{L}{0.4\textwidth}
  \vspace{-20pt}
  \begin{center}
    \includegraphics[page=1,width=0.99\textwidth]{./images/student}
  \end{center}
  \vspace{-40pt}
\end{wrapfigure}


\begin{wrapfigure}{L}{0.4\textwidth}
  \vspace{-20pt}
  \begin{center}
    \includegraphics[page=2,width=0.99\textwidth]{./images/student}
  \end{center}
  \vspace{-40pt}
\end{wrapfigure}


\begin{wrapfigure}{L}{0.4\textwidth}
  \vspace{-20pt}
  \begin{center}
    \includegraphics[page=3,width=0.99\textwidth]{./images/student}
  \end{center}
  \vspace{-40pt}
\end{wrapfigure}

\clearpage
\section{Prototypen}
\label{sec:prototypen}

(3 Seiten)

Lorem ipsum dolor sit amet, consectetur adipiscing elit. Proin dolor nulla, accumsan non imperdiet convallis, congue nec orci. Nulla id nunc arcu. Fusce a congue metus. Etiam ex nunc, egestas ut urna vitae, commodo ultrices neque. Suspendisse id quam ut nulla sagittis laoreet ut quis nulla. Proin mollis vitae tortor non dignissim. Proin sed nulla eu dolor mattis auctor. Vestibulum eleifend interdum ligula eget pharetra. Integer sollicitudin non arcu non aliquam. Cras placerat ante at pretium vulputate.

Cum sociis natoque penatibus et magnis dis parturient montes, nascetur ridiculus mus. Cras vel augue molestie magna auctor convallis. Nullam tincidunt pharetra orci. Class aptent taciti sociosqu ad litora torquent per conubia nostra, per inceptos himenaeos. Vestibulum congue risus orci, ac accumsan ante pretium et. Vestibulum maximus massa vitae sodales convallis. Integer sed mollis metus, eu porta

\clearpage
\section{Vorherige Probetests}
\label{sec:probetests}

(2 Seiten)

Lorem ipsum dolor sit amet, consectetur adipiscing elit. Proin dolor nulla, accumsan non imperdiet convallis, congue nec orci. Nulla id nunc arcu. Fusce a congue metus. Etiam ex nunc, egestas ut urna vitae, commodo ultrices neque. Suspendisse id quam ut nulla sagittis laoreet ut quis nulla. Proin mollis vitae tortor non dignissim. Proin sed nulla eu dolor mattis auctor. Vestibulum eleifend interdum ligula eget pharetra. Integer sollicitudin non arcu non aliquam. Cras placerat ante at pretium vulputate.

Cum sociis natoque penatibus et magnis dis parturient montes, nascetur ridiculus mus. Cras vel augue molestie magna auctor convallis. Nullam tincidunt pharetra orci. Class aptent taciti sociosqu ad litora torquent per conubia nostra, per inceptos himenaeos. Vestibulum congue risus orci, ac accumsan ante pretium et. Vestibulum maximus massa vitae sodales convallis. Integer sed mollis metus, eu porta

\clearpage
\section{Testbericht}
\label{sec:testbericht}

\subsection{Im Test erkannte Probleme}
\label{sec:foundproblemes}

Durch den Test der Android App sind einge Schwächen unseres Prototypen aufgefallen. Besonders hervorzuheben ist hier die Struktur der App.

Beispielsweise ist bereits zu Beginn aufgefallen, dass dem Probanden nicht klar war, was auf dem Homescreen dargestellt wird und was nicht. Wir hatten uns dazu entschlossen, auch bereits abgelaufene Umfragen unter den aktuellen Umfragen aufzulisten, was offensichtlich zur Verwirrung geführt hat. Somit war es für den Probanden nicht klar, dass wir einen seperaten Screen für die von ihm abgegebenen Argumente haben, was ja quasi einem Archiv entspricht.

Als nächstes stellten wir fest, dass der Proband Schwierigkeiten damit hatte die globalen Einstellungen von den Ansichtseinstellungen des Homescreens zu unterscheiden. Gemeint sind hier die Einstellungen zum Ausblenden von bereits abgelaufenen Umfragen und der Sortierung aller Umfragen. Der Button, der dieses Menü aufruft wurde erst sehr spät gefunden.


\subsection{Abgeleitete Verbesserungsvorschläge}
\label{sec:verbesserungsvorschlaege}

Man kann aus den im Test festgestellten Problemen einige Verbesserungsvorschläge ableiten. Zum Einen sollte man definitiv die Struktur der App überdenken. Der Homescreen sollte die bereits abgelaufenen Umfragen nicht weiter anzeigen, sondern in einen gesonderten Screen auslagern, den man als Archiv bezeichnen kann. Hierbei ist dann noch wichtig, dass eine Differenzierung der bereits abgelaufenen Umfragen und der von dem User abgegebenen Argumente zu allen (auch abgelaufenen) Umfragen notwendig ist. Wie man diese Teile am besten trennt müsste durch weitere Tests herausgefunden werden. Dadurch, dass diese Funktionalität jedoch vermutlich nur selten genutzt wird und daher keine wichtge Funktion ist, sollte man diese Screens in das Slide-In Menü integrieren, da dieses Menü primär genutzt wird, wenn man erweiterte Funktionen benötigt.

Außerdem sollte man die Sortierung der Umfragen in die bereits existierenden Einstellungen auslagern, da die App somit noch simpler wird.

\subsection{Kritik und Bewertung des Tests}
\label{sec:kritik}

Beim Test unseres Prototypen sind uns einige Dinge aufgefallen, die primär konzeptioneller Natur waren. Es hat bereits damit angefangen, dass wir durch technische Probleme nicht alle Dokumente ausdrucken konnten und somit dann auch zu spät zu dem verabredeten Test gekommen sind. Daraus entstand dann eine entsprechend gestresste Situation, die man durch bessere Planung hätte umgehen können. 
Nachdem wir dann alles für den Test gerichtet hatten viel relativ schnell auf, dass wir unsere App zu ausführlich testen wollten und dass wir uns nicht auf spezielle Fälle fokussierten. Wir wollten sowohl die Webapplikation, als auch die Android App testen und hatten dafür sehr viele Testfälle vorbereit, was auch dem Testleiter bewusst war. Daraus resultierte, dass die Testvorbesprechung viel zu kurz kam und eventuelle Fragen eventuell nicht geklärt wurden.
Beim eigentlichen Test haben wir dann festgestellt, dass die eigentlichen Funktionen erst viel zu spät dran kamen, so hatten wir als erste Fälle gefragt, wie man Einstellungen ändert und das Archiv findet und daraus Informationen ableitet. Es wäre an dieser Stelle wesentlich sinnvoller gewesen, wenn man zuerst die Abfrage von Informationen zu den existierenden Umfragen gefordert hätte, da sich der Proband dann auch schneller an den strukturellen Aufbau der App hätte gewöhnen können und daraus resultierend die Aufgaben schneller hätte lösen können. Durch unsere Fragen verwirrten wir den Probanden jedoch massiv, was dem Test alles andere als förderlich war.

Wie bereits angesprochen, haben wir die Testdauer massivst unterschätzt, so kam es dazu, dass wir Testaufgaben überspringen mussten und der Prototyp für die Webapp überhaupt nicht getestet wurde. In Zukunft ist eine bessere Abschätzung des Verlaufs von essentieller Bedeutung.

Auch wenn nicht alles absolut perfekt funktioniert hat, kann man festhalten, dass wir durch die gemachten Fehler viel gelernt haben, was man in Zukunft vermeiden kann.

\clearpage